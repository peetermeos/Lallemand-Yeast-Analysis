\section{Recommendations}
Having completed the initial analysis of the yeast production process we have the following recommendations for further process improvements and better analytical insight:

\begin{itemize}
\item Digitize and automate if possible the current manual data collection on the production batches that is at present time in a form of scanned PDF files. The current system is error prone and does not lend itself easily for analytics.
\item Consider automatic loading of process data that is currently in CSV files into a time series database. That will provide for long term data storage and ease the preparatory steps for any future analytics on the fermentation process.
\end{itemize}

\subsection{Data Collection}
\subsubsection{Is every recorded parameter labelled? }

\emph{(i.e. which measurements correspond to ‘Rejected’ and which ones correspond to ’Released’ statuses?  which measurements belongs to ‘Seed’ number 4151 and which ones to ‘Seed’ number 5292?)}

To analyze the parameters collected from the yeast cultivation process online (and offline) and identifying the deviations that cause quality problems, it requires enough historical data that allows to capture information about exact batch numbers and statuses wrt. time and production cycle. 

Recommend to automate collection of status information such as:
\begin{itemize}
    \item Production Date (MM-DD-YYYY)
    \item Strain
    \item Seed (batch number)
    \item Commercial
    \item Lot no
    \item Status - quality control decision: Rejected, Released, Restricted release
\end{itemize}

It needed to be matched with time-series data generated by yeast cultivation and fermentation process.  

\subsubsection{Is every time-stamp information for each observation grouped by cycles?}

As many observations belongs to the one production cycle (M:1) that means approx. 3 days durability for the one particular seed/batch, so after matching the data it should not have \emph{gaps or overlaps} in time period.  

Proposed new records to process Row Data: 
\begin{itemize}
    \item production line active / not active
    \item new cycle start point / no change
    \item cycle standing / running (cycle serial number has to be assigned) 
\end{itemize}

\subsubsection{Paperless Data Collection Consideration}
\begin{itemize}
    \item Manual fermentation sheets ("7442ferm sheets"), containing manually measured values of the pH, spin test and other necessary variables - they are \emph{not currently sufficient for analysis} and it require to \emph{collect the history in a digital way}. 
    \item Making some fields ‘mandatory’ helps to submit data without missing it and increase Data Completeness (whether there are any gaps in the data from what was expected to be collected, and what was actually collected).
    \item Make sure Data Integrity (the data has not been changed when performing any operation on them, whether it is transfer, storage, display or during data preparation).
\end{itemize}

\subsection{Data Preparation}
To improve data preparation historical data instead of CSV files could be stored in some database tables or at least \emph{pre-merged files corresponding to original timestamp and business process step}.

The main idea here is to have  precise mapping of date and time along the whole process to be able to monitor 1 batch from start to end like a holistic process workflow (for example to merge events from F1>F2>F4 steps vertically or horizontally, following and keeping  the full production time interval (approx. 3 days) required for producing the one yeast seed /batch.) 

Yeast separation, storage container data could be also referred to batch, seed and production dates. 
\subsection{Data Analysis}

\begin{itemize}
    \item	Having more samples with labelled observations (Strain, Seed, Status) give higher probability to identify deviations that cause quality problem, identify which variables measurements impact quality and predict the production result /status.
    \item Having paperless recorded variables (i.e. pH) allow to analyze collected parameters more holistically 
\end{itemize}