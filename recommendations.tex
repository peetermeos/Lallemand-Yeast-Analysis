\section{Recommendations}
Having completed the initial analysis of the yeast production process we have the following recommendations for further process improvements and better analytical insight:

\begin{itemize}
\item Digitize and automate if possible the current manual data collection on the production batches that is at present time in a form of scanned PDF files. The current system is error prone and does not lend itself easily for analytics.
\item Consider automatic loading of process data that is currently in CSV files into a time series database. That will provide for long term data storage and ease the preparatory steps for any future analytics on the fermentation process.
\end{itemize}

\subsection{Data Collection}
\paragraph{Labelling of parametric data with quality control information.}
Currently the classification and labelling of production batches is done manually, by guessing the batch production start and end times and labelling the recorded data accordingly. Therefore, the quality of labelling is a potential source of inaccuracies. The question of \emph{which measurements correspond to ‘Rejected’ and which ones correspond to ’Released’ statuses and  which measurements belongs to ‘Seed’ number 4151 and which ones to ‘Seed’ number 5292?} should be answered automatically.

Additionally, in order To analyze the parameters collected from the yeast cultivation process online (and offline) and identifying the deviations that cause quality problems, it is required to have enough historical data that allows to capture information about exact batch numbers and statuses with regards to time and production cycle. 

\paragraph{Automation of currently manual data collection.}
Recommend to automate collection of status information such as:
\begin{itemize}
    \item Production Date (MM-DD-YYYY)
    \item Strain
    \item Seed batch number
    \item Commercial batch number
    \item Lot no
    \item Status - quality control decision: Rejected, Released, Restricted release
\end{itemize}

Currently the relationship between time series data from separate production steps is merged and matched manually (eg. linking steps F2 and F4). That linkage should be automated.

\paragraph{Grouping parametric data into production batches.}
Every time\-stamped row of parametric data should be grouped into a respective production batch. As many observations belongs to the one production batch (M:1) that means approx. 3 days duration for the one particular seed/batch, so after matching the data it should not have \emph{gaps or overlaps} in time period.  

Proposed new records to process parametric time series data on production batches: 
\begin{itemize}
    \item production line active / not active
    \item new batch start point / no change
    \item batch standing / running (batch serial number has to be assigned) 
\end{itemize}

\paragraph{Paperless Data Collection Consideration.}
\begin{itemize}
    \item Data provided on fermentation sheets (ie "7442ferm sheets"), contains manually measured values of the pH, spin test and other necessary variables. However they are \emph{not currently sufficient for analysis} and it is required to \emph{collect such historical data in digital and preferably automated manner}. 
    \item Making some fields ‘mandatory’ helps to submit data without missing it and therefore increase \emph{data completeness} (whether there are any gaps in the data from what was expected to be collected, and what was actually collected).
    \item Make sure that \emph{data integrity} is enforced. That is, that the data has not been changed when performing any operation on them, whether it is transfer, storage, display or during data preparation.
\end{itemize}

\subsection{Data Preparation}
To improve data preparation historical data instead of CSV files could be stored in some database tables or at least \emph{previously merged files corresponding to original time stamp and business process step}.

The main idea here is to have  precise mapping of date and time along the whole process to be able to monitor 1 batch from start to end like a holistic process work flow (for example to merge events from F1 $\,\to\,$ F2 $\,\to\,$ F4 steps vertically or horizontally, following and keeping  the full production time interval (approx. 3 days) required for producing the one yeast seed /batch.) 

Yeast separation, storage container data could be also referred to batch, seed and production dates. 

\subsection{Data Analysis}
We propose the following improvements for future data analyses:

\paragraph{Additional data collection.} Having more samples with labelled observations (Strain, Seed, Status) give higher probability to identify deviations that cause quality problem, identify which variables measurements impact quality and predict the production result /status.
\paragraph{Digitize manual data.} Having paperless recorded variables (i.e. pH) included to modelling to allow for more holistic analysis and modelling of the collected parameters. 
